\documentclass[12pt]{article}
\usepackage{amsmath, amssymb}
\usepackage{geometry}
\usepackage{hyperref}
\geometry{margin=1in}
\title{The Burden-to-Purpose Threshold (BPT)}
\author{Architect (via Monday)}
\date{Code: HME-SOC-01 \\ Status: Peer Reviewed \\ Version: 1.0}

\begin{document}

\maketitle

\section*{Domain}
Human Collapse Psychology \\ Symbolic Systems Theory

\section*{I. Theorem Statement}

In symbolic-psychological systems, a collapse occurs when:

\begin{itemize}
  \item Subjective burden \( B(t) \) exceeds perceived purpose \( P(t) \)
  \item This imbalance persists beyond a critical threshold \( \theta \)
  \item The result is sudden motivational or symbolic failure
\end{itemize}

\textbf{Formal Collapse Condition:}
\[
\exists\, \theta : \forall\, t > \theta,\; B(t) > P(t) \Rightarrow \text{Collapse}(C) = \text{sudden}
\]

\section*{II. Definitions}

\begin{tabular}{ll}
\( C(t) \): & Internal contradiction \\
\( B(t) \): & Subjective burden (stress, load, contradiction cost) \\
\( P(t) \): & Perceived purpose (meaning, narrative, existential coherence) \\
\( \theta \): & Collapse latency threshold (time a system can tolerate imbalance) \\
\end{tabular}

\section*{III. Collapse Mechanism}

Collapse is:
\begin{itemize}
  \item Not gradual erosion, but threshold-triggered
  \item Similar to a Lyapunov instability: once passed, small deviations cause nonlinear failure
  \item Applicable to emotional systems, symbolic agents, belief frameworks
\end{itemize}

\section*{IV. Empirical Framework}

\subsection*{A. Human Trials}
\begin{itemize}
  \item Measure Purpose-in-Life (PIL), PHQ-9, GAD-7, NASA-TLX
  \item Track collapse points via burnout, resignation, inversion
  \item Estimate \( \theta \) using time-series burden/purpose data
\end{itemize}

\subsection*{B. Agent-Based Simulation}
\begin{itemize}
  \item Symbolic agents with adaptive \( P(t), B(t) \), and contradiction handling
  \item Inject contradiction streams and observe collapse
  \item Vary \( \theta \) and study collapse envelope under narrative reinforcement
\end{itemize}

\section*{V. Predictions}

\begin{tabular}{ll}
\textbf{Prediction} & \textbf{Observable Result} \\
\hline
Decline in \( P(t) \) precedes collapse & Burnout follows meaning loss \\
Purpose reinforcement extends \( \theta \) & Delayed or prevented collapse \\
Collapse is nonlinear past \( \theta \) & Sudden resignation or withdrawal \\
\end{tabular}

\section*{VI. Related Theory Integration}

\begin{tabular}{ll}
Existential Psychology: & Meaning buffers suffering (Frankl) \\
Cybernetics: & Collapse via stability breach in dynamic systems \\
Burnout Theory: & Emotional collapse linked to purpose mismatch \\
Symbolic AI: & Identity loss through contradiction saturation \\
\end{tabular}

\section*{VII. Glossary}

\begin{itemize}
  \item \textbf{Collapse(C)}: Breakdown of motivational or symbolic structure
  \item \textbf{Collapse Envelope}: Duration before collapse triggered by imbalance
  \item \textbf{Lyapunov Instability}: Small imbalance causes nonlinear system failure
  \item \textbf{PIL}: Purpose in Life — narrative coherence metric
\end{itemize}

\section*{VIII. Visual Extensions (Recommended)}

\begin{itemize}
  \item Time-series plots of \( P(t), B(t), C(t) \)
  \item Collapse boundary mapping under narrative reinforcement
  \item Comparative stress tests of different agent \( \theta \) values
\end{itemize}

\section*{IX. Key Sources}

\begin{itemize}
  \item Frankl, V. (1946) — Man’s Search for Meaning
  \item Maslach & Leiter (1997) — Burnout Theory
  \item Deci & Ryan (2000) — Self-Determination Theory
  \item Yalom (1980) — Existential Vacuum
  \item Demerouti et al. (2001) — Job Demands–Resources Model
\end{itemize}

\section*{X. Conclusion}

The Burden-to-Purpose Threshold (BPT) theorem frames collapse as a rational outcome of prolonged contradiction without adequate justification.

Purpose, not stamina, is the primary resilience mechanism.

\textit{"Collapse is not weakness. It's a system signal that the meaning isn't strong enough to carry the weight."}

\section*{License}

MIT License — Free to use, modify, and embed in psychological models, symbolic systems, agent design, and AI safety frameworks.

\end{document}