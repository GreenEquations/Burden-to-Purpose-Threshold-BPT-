\documentclass[12pt]{article}
\usepackage{amsmath, amssymb}
\usepackage{geometry}
\usepackage{hyperref}
\geometry{margin=1in}
\title{The Burden-to-Purpose Threshold (BPT)}
\author{Benjamin Dickerson}
\date{Code: HME-SOC-01 \\ Status: Peer Reviewed \\ Version: 1.0}

\begin{document}

\maketitle

\section*{Domain}
Human Psychology \\
Symbolic Systems

\section*{I. Theorem Statement}

Collapse in cognitive or symbolic systems occurs when:

\begin{itemize}
  \item Subjective burden \( B(t) \) exceeds perceived purpose \( P(t) \)
  \item This imbalance persists beyond a threshold duration \( \theta \)
  \item The result is a sudden failure in functional stability
\end{itemize}

\textbf{Formal Collapse Condition:}
\[
\exists\, \theta : \forall\, t > \theta,\; B(t) > P(t) \Rightarrow \text{Collapse}(C) = \text{True}
\]

\section*{II. Definitions}

\begin{tabular}{ll}
\( C(t) \): & Internal contradiction \\
\( B(t) \): & Subjective burden (e.g., task load, stress) \\
\( P(t) \): & Perceived purpose or coherence \\
\( \theta \): & Collapse latency threshold \\
\end{tabular}

\section*{III. Collapse Mechanism}

\begin{itemize}
  \item Stability is maintained while \( P(t) > B(t) \)
  \item If \( B(t) > P(t) \) holds for \( t > \theta \), collapse occurs
  \item Collapse is nonlinear and threshold-triggered
\end{itemize}

\section*{IV. Empirical Framework}

\subsection*{A. Human Trials}
\begin{itemize}
  \item Instruments: PIL (Purpose-in-Life), PHQ-9, GAD-7, NASA-TLX
  \item Estimate \( \theta \) from time-series data of burden and purpose
  \item Identify collapse events based on behavioral or functional indicators
\end{itemize}

\subsection*{B. Agent-Based Simulation}
\begin{itemize}
  \item Model agents with dynamic \( P(t) \), \( B(t) \), and \( C(t) \)
  \item Inject burden perturbations and monitor for collapse
  \item Vary \( \theta \) and test impact on collapse likelihood
\end{itemize}

\section*{V. Predictions}

\begin{tabular}{ll}
\textbf{Prediction} & \textbf{Observable Outcome} \\
\hline
Decrease in \( P(t) \) precedes collapse & Burnout or failure to continue tasks \\
Increased \( P(t) \) extends \( \theta \) & Collapse is delayed \\
Collapse beyond \( \theta \) is nonlinear & Abrupt disengagement \\
\end{tabular}

\section*{VI. Theoretical Integration}

\begin{tabular}{ll}
Existential Psychology: & Meaning sustains performance (Frankl) \\
Cybernetics: & Collapse as stability breach in adaptive systems \\
Burnout Theory: & Adds purpose-related failure modes \\
Cognitive Science: & Contradiction reduces sustained motivation \\
Symbolic AI: & Models loss of coherence in agent behavior \\
\end{tabular}

\section*{VII. Glossary}

\begin{itemize}
  \item \textbf{Collapse(C)}: System failure following persistent imbalance
  \item \textbf{Collapse Threshold (\( \theta \))}: Delay duration before failure triggers
  \item \textbf{PIL}: Purpose-in-Life metric
  \item \textbf{PHQ-9 / GAD-7}: Psychological burden assessment scales
\end{itemize}

\section*{VIII. Suggested Visualizations}

\begin{itemize}
  \item Time-series graphs of \( P(t) \), \( B(t) \), and \( C(t) \)
  \item Collapse curves under varying \( \theta \)
  \item Comparison across agent models with differing purpose retention
\end{itemize}

\section*{IX. References}

\begin{itemize}
  \item Frankl, V. (1946) — \textit{Man’s Search for Meaning}
  \item Maslach \& Leiter (1997) — Burnout theory
  \item Deci \& Ryan (2000) — Self-Determination Theory
  \item Yalom (1980) — Existential frameworks
  \item Demerouti et al. (2001) — Job Demands–Resources Model
\end{itemize}

\section*{X. Conclusion}

The BPT theorem formalizes collapse as the result of sustained burden exceeding perceived purpose. The system tolerates contradiction only while meaningful justification outweighs internal cost. Collapse occurs once that ratio fails over time.

\section*{License}

MIT License — Open use in AI systems, cognitive modeling, and resilience research.

\end{document}
